\documentclass{article}

% Packages for setting up page margins
\usepackage[margin=1in]{geometry}

\usepackage{graphicx, setspace, amsmath, mathtools, amssymb}

% Title
\title{CS536 Science of Programming - Assignment 6}
\author{Batkhishig Dulamsurankhor - A20543498}
\date{\today} % Use \date{} for no date

\begin{document}

\maketitle


\section*{Problem 1}

Postcondition $x=fac(n)$, precondition $n \geq 0$. Create a loop invariant $p$ by replacing $n$ by variable $y$ in the postcondition.
\vspace{10pt}

If we replace $n$ by variable $y$, we get $x=fac(y)$. Because we need the factorial of the first $n$ natural numbers, we have to intialize $y=1$ and increment it on each iteration until $y=n$.
\vspace{10pt}

$\{\textbf{inv }p \equiv x=fac(y) \wedge 1 \leq y \leq n \}\{\textbf{bd } n-y\}$

$ \textbf{while } y \neq n \textbf{ do }$

\qquad \qquad ... make y larger ...

$\textbf{ od}$

$\{x=fac(y) \wedge 1 \leq y \leq n \wedge y=n\}$

$\{x=fac(n)\}$


\section*{Problem 2}

Create full proof outline under the total correctness.
\vspace{10pt}

Let's consider precondition of the loop. From loop invariant, we know that $1 \leq y \leq n$, so that it is logical to start the loop with $y=1$. $x$ must also be $x=1$. If one of them starts with $0$, then the value of $x$ will never be other than $0$.
\vspace{10pt}

$\{x=1 \wedge n \geq 1 \wedge y=1 \}$

$\{\textbf{inv }p \equiv x=fac(y) \wedge 1 \leq y \leq n \}\{\textbf{bd } n-y\}$

$ \textbf{while } y \neq n \textbf{ do }$

\qquad \qquad ... make y larger ...

$\textbf{ od}$

$\{x=fac(y) \wedge 1 \leq y \leq n \wedge y=n\}$

$\{x=fac(n)\}$

\vspace{10pt}
Loop body.
\vspace{10pt}

$\{x=1 \wedge n \geq 1 \wedge y=1 \}$

$\{\textbf{inv }p \equiv x=fac(y) \wedge 1 \leq y \leq n \}\{\textbf{bd } n-y\}$

$ \textbf{while } y \neq n \textbf{ do }$

\qquad \qquad $x \coloneqq x*(y+1); y \coloneqq y+1;$

$\textbf{ od}$

$\{x=fac(y) \wedge 1 \leq y \leq n \wedge y=n\}$

$\{x=fac(n)\}$


\section*{Problem 3}

True or False.

\vspace{10pt}

a. True. We aim for stronger postcondition.

b. False. Although the statement is the definition of a loop invariant, it doesn't imply a good loop invariant. $p$ can be $p \equiv T$ but it is not a good one.

c. False. There is no algorithm to find bound expressions.

d. 

e.


\section*{Problem 4}

Full proof outline with forward assignment.

\vspace{10pt}

First, inner array substitution:

$ (b[i])[k/b[j]] \equiv \textbf{if } (i=j) \textbf{ then } k \textbf{ else } b[i] \textbf{ fi} $

\vspace{10pt}
Then, outer array substitution:

$ (b[b[i]])[k/b[j]] $

\qquad \qquad $ \equiv \textbf{if } ((\textbf{if } (i=j) \textbf{ then } k \textbf{ else } b[i] \textbf{ fi})=j) \textbf{ then } k \textbf{ else } b[\textbf{if } (i=j) \textbf{ then } k \textbf{ else } b[i] \textbf{ fi}] \textbf{ fi} $

\qquad \qquad $ \mapsto  \textbf{if } (\textbf{if } (i=j) \textbf{ then } k=j \textbf{ else } b[i]=j \textbf{ fi}) \textbf{ then } k \textbf{ else } b[\textbf{if } (i=j) \textbf{ then } k \textbf{ else } b[i] \textbf{ fi}] \textbf{ fi} $

\qquad \qquad $ \mapsto  \textbf{if } (i=j \wedge k=j) \vee (i \neq j \wedge b[i]=j) \textbf{ then } k \textbf{ else } b[\textbf{if } (i=j) \textbf{ then } k \textbf{ else } b[i] \textbf{ fi}] \textbf{ fi} $

\qquad \qquad $ \mapsto  \textbf{if } (i=j \wedge k=j) \vee (i \neq j \wedge b[i]=j) \textbf{ then } k \textbf{ else } (\textbf{if } (i=j) \textbf{ then } b[k] \textbf{ else } b[b[i]] \textbf{ fi}) \textbf{ fi} $

\qquad \qquad $ \mapsto  \textbf{if } (i=j \wedge k=j) \vee (i \neq j \wedge b[i]=j) \textbf{ then } k \textbf{ else if } (i=j) \textbf{ then } b[k] \textbf{ else } b[b[i]] \textbf{ fi} $

\section*{Problem 5}

Find an optimized precondition $p$ and create full proof outline: $\{p\}b[i] \coloneqq x;b[j] \coloneqq y \{b[i] \leq b[j]\}$.

\vspace{10pt}

$ wp(b[i] \coloneqq x;b[j] \coloneqq y, b[i] \leq b[j]) $

\qquad \qquad $ \equiv wp(b[i] \coloneqq x, wp(b[j] \coloneqq y, b[i] \leq b[j]))$

\qquad \qquad $ \equiv wp(b[i] \coloneqq x, (b[i] \leq b[j]) [y/b[j]])$

\qquad \qquad $ \equiv wp(b[i] \coloneqq x, (b[i])[y/b[j]] \leq (b[j])[y/b[j]])$

\qquad \qquad $ \equiv wp(b[i] \coloneqq x, (\textbf{if } (i=j) \textbf{ then } y \textbf{ else } b[i] \textbf{ fi}) \leq (\textbf{if } (j=j) \textbf{ then } y \textbf{ else } b[j] \textbf{ fi}))$

\qquad \qquad $ \mapsto wp(b[i] \coloneqq x, (\textbf{if } (i=j) \textbf{ then } y \textbf{ else } b[i] \textbf{ fi}) \leq y)$

\qquad \qquad $ \mapsto wp(b[i] \coloneqq x, \textbf{if } (i=j) \textbf{ then } y \leq y \textbf{ else } b[i] \leq y \textbf{ fi})$

\qquad \qquad $ \mapsto wp(b[i] \coloneqq x, \textbf{if } (i=j) \textbf{ then } T \textbf{ else } b[i] \leq y \textbf{ fi})$

\qquad \qquad $ \mapsto wp(b[i] \coloneqq x, (i=j) \vee (b[i] \leq y))$

\qquad \qquad $ \equiv (i=j \vee (b[i] \leq y))[x/ b[i]]$

\qquad \qquad $ \equiv i=j \vee (b[i])[x/ b[i]] \leq y$

\qquad \qquad $ \equiv i=j \vee (\textbf{if } (i=i) \textbf{ then } x \textbf{ else } b[i] \leq y \textbf{ fi}) \leq y$

\qquad \qquad $ \mapsto i=j \vee x \leq y$

\vspace{10pt}

A valid triple: $ \{i=j \vee x \leq y\}b[i] \coloneqq x;b[j] \coloneqq y \{b[i] \leq b[j]\}$.

\vspace{10pt}

$ \{i=j \vee x \leq y\}b[i] \coloneqq x;b[j] \coloneqq y \{b[i] \leq b[j]\}$

\qquad \qquad $ \equiv \{i=j \vee x \leq y\}b[i] \coloneqq x \{b[i] \leq y\} b[j] \coloneqq y \{b[i] \leq b[j]\}$ \qquad \qquad // backward assignment

\qquad \qquad $ \equiv \{i=j \vee x \leq y\}\{x \leq y\} b[i] \coloneqq x \{b[i] \leq y\} b[j] \coloneqq y \{b[i] \leq b[j]\}$ \qquad // backward assignment

\end{document}